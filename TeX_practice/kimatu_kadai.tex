\documentclass{jarticle}
\begin{document}

\title{金沢研期末課題\\SIRモデルによる感染症流行の分析}
\author{72005560田村太陽}
\maketitle

\begin{flushleft}\
2019年12月以降世界では新型コロナウイルス感染症が流行している.そのような状況下では,適切な政策・対策を講じていくには感染症数理モデルを用いた流行の分析が必要不可欠である.本レポートでは,人から人へ直接伝播する感染症の流行動態をとらえた基本的な数理モデルである,SIRモデルについて述べる.
\end{flushleft}


SIRモデルについて\\
SIRモデルとは前述したように,人から人へ直接伝播する感染症の流行動態をとらえた基本的な数理モデルのことであり,S・I・Rはそれぞれ,免疫のない非感染個体(Susceptible)・感染個体(Infectious)・回復個体(Recovered)を表す.この数理モデルは1927年にKermackとMcKendrickの論文で初めて導入され,彼らは、時刻$t$における免疫のない非感染個体$S(t)$と感染個体数$I(t)$と回復個体数$R(t)$の3つの未知関数の時間変化に伴う増減に着目して、次の常微分方程式で記述されるとした.\\
\begin{equation}
\frac{dS(t)}{dt} = -\beta S(t)I(t) 
\end{equation}
\begin{equation}
\frac{dI(t)}{dt} = -\beta S(t)I(t)-\gamma I(t)
\end{equation}
\begin{equation}
\frac{dR(t)}{dt} = \gamma I(t)
\end{equation}
なお,総人口は1とし,$S(t) + I(t) + R(t) = 1$である.
ここで,$\beta$は単位時間当たりの感染率を表す係数であり,$\beta I(t)$は時間$t$における感染力と定義される.つまり,人口が一定である場合,感染力は集団内の感染者数に比例することがわかる.また,$\gamma$は単位時間当たりの回復や隔離による除去率を表す係数であり,逆数$\gamma ^{-1}$はは感染から回復もしくは隔離されるまでの平均感染性期間である.なおこの系では,短期的な流行を考慮しているモデルであるため,出生や・死亡などの人口動態はほぼ無視できるものと仮定している.\\
式(1)について左辺は,免疫のない非感染個体数S(t)の変化の割合を表していて,それが右辺に現れる非感染個体数S(t)と感染個体数I(t)の積に比例して減少することを表している.つまり,感染していない個体の数は,感染している個体との接触により減少するということである.\\
式(2)について左辺は,感染個体数I(t)の変化の割合を表していて,それが右辺のように 非感染個体数S(t)と感染個体数I(t)の積に比例して増加しつつ,感染個体数I(t)に比例して一定の割合で減少することを表している.つまり,感染している個体の数は,感染していない個体との接触により増加しつつ,一定の割合で回復するから減少するということである.\\
式(3)について左辺は,回復個体数R(t)の変化の割合を表していて,それが感染個体数に比例して一定の割合で増加することを表している.つまり,感染者は一定の割合で回復するため,その分回復者数が増えるということでる.\\
ここで,式(2)を変形すると次の式(4)が得られる.\\
\begin{equation}
\frac{dI(t)}{dt} = (\beta S(t)-\gamma) I(t)
\end{equation}
この右辺の$\beta S(t)-\gamma$が$\beta S(t)-\gamma >0$となる場合は,新規感染者が増加している状況であり,感染流行の条件でもある.またこの式を,$\frac{\beta S(t)}{\gamma} > 1$と変形すると,時刻0においては集団の全員が免疫のない非感染個体であるため,$S(0) = 1$とおくことができ,$\frac{\beta}{\gamma}$が感染症流行の閾値となる.そして,この値は基本再生産数($R_{0}$)と呼ばれている.\\
基本再生産数($R_{0}$)は,ある感染症に対して全員が感受性を持つ集団のなかで,典型的な1人の感染者が感染性を有する期間に再生産する二次感染者数の平均値とされる.$R_{0}$が1より大きい場合は大規模な流行が発生しうるが,1より小さい場合は流行は自然消滅する.\\
感染症対策下における再生産数は,実効再生産数と呼ばれ,実効再生産数を1以下にすることが政策・対策における1つの指標となる.実効再生産数は以下の式(5)で表される.
\begin{equation}
R = (1-q) R_{0}
\end{equation}
ここで$q$は,マスク着用・手洗い・接触削減といった感染対策による感染力の低下率を表している.また,ワクチン接種による感染予防効果としても考えることができ,これを$p$とすると,ワクチン接種による感染症根絶のための条件は,
\begin{equation}
(1-p) R_{0} < 1
\end{equation}
である.この式(6)を$p$について解いたときの割合$1-\frac{1}{R_{0}}$は,臨界免疫割合といい,ワクチンの効果が明らかであれば$R_{0}$に従いワクチン接種率目標を大まかに設定するうえで役に立つ.

\begin{thebibliography}{99}
  \item 西浦博; 稲葉寿. 感染症流行の予測: 感染症数理モデルにおける定量的課題. 2006.
  \item 西浦博; 鈴木絢子.感染症の数理モデルと対策. 2020
\end{thebibliography}
\end{document}